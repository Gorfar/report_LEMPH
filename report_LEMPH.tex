\documentclass[a4paper,12pt]{article}
\usepackage[T2A]{fontenc}
\usepackage[utf8]{inputenc}
\usepackage[russian]{babel}
\usepackage{amssymb,amsmath}
\usepackage {indentfirst}
\textheight=24cm % высота текста
\textwidth=16cm % ширина текста
\oddsidemargin=0pt % отступ от левого края
\topmargin=-1.5cm % отступ от верхнего края
\usepackage[pdftex]{graphicx}
\graphicspath{{noiseimages/}}
\tolerance = 2000
\parindent = 24pt
\begin{document}
\begin{titlepage}
\begin{center}


\textsc{\LARGE University of Beer}\\[1.5cm]

\textsc{\Large Final year project}\\[0.5cm]

% Title
\HRule \\[0.4cm]
{ \huge  Отчет о семестровой работе \\
Название работы
\\[0.4cm] }

\HRule \\[1.5cm]

% Author and supervisor

\begin{flushright} \large
\emph{Работу выполнили:} \\
Ушаков Александр, 125 гр. \\
Храмцов Игорь, 125 гр. \\
\emph{Научный руководитель:} \\
Миланич А.И.
\end{flushright}


\vfill

% Bottom of the page
{\large \today}

\end{center}
\end{titlepage}
\tableofcontents
\newpage
\section{Введение}
\subsection{Теоретическая справка}

\textbf{Рецепторы глаза} (фоторецепторы) являются специализированным типом нейронов (нервных клеток), обнаруженных в сетчатке глаза, которые способны к формированию электрического сигнала в ответ на внешний фотостимул (фототрансдукция). Фоторецепторы содержатся во внешнем зернистом слое сетчатки. Фоторецепторы отвечают гиперполяризацией (а не деполяризацией, как другие нейроны) в ответ на адекватный этим рецепторам сигнал — свет. Фоторецепторы размещаются в сетчатке очень плотно, в виде шестиугольников (гексагональная упаковка). 


К фоторецепторам относят палочки и колбочки, существуют функциональные различия между ними. Колбочки чувствительны ко всей видимой области спектра и при достаточном освещении позволяют глазу ощущать и анализировать различные цвета и оттенки; палочки же — более чувствительны, но их спектр и максимум чувствительности меняются в зависимости от яркости освещения. При более ярком свете спектр чувствительности палочек, благодаря содержащимся в них фотопигменте родопсине, сдвинут в коротковолновую (синюю) область спектра, а при слабом, сумеречном освещении, их чувствительность возрастает, при этом максимум чувствительности палочек смещается в зелёную область спектра. Это свойство палочек и вызывает эффект Пуркинье, проявляющийся в искажении цветов при переходе от дневного зрения к сумеречному. В сумерках и ночью, без участия колбочек, палочки не могут анализировать цвета и поэтому при слабом освещении глаза цвета не различает.


\textbf{Трехкомпонентная теория зрения.} 
По мнению сторонников трёхкомпонентной гипотезы зрения, раз найдены три пика поглощения в видимой области спектра тканями сетчатки, то это должно обуславливаться наличием трёх типов зрительных пигментов и, как они считают, должны существовать три вида колбочек, чувствительных к разным длинам волн света (цветам). Предполагается наличие колбочек S-типа чувствительных в синей (S от англ. Short — коротковолновый спектр), M-типа — в зеленой (M от англ. Medium — средневолновый), и L-типа — красной (L от англ. Long — длинноволновый) частях спектра. При этом исходят из предположения, что в каждом типе колбочек содержится только один из трёх пигментов. На сегодняшний день эти предположения подтвердить так и не удалось.


В настоящее время известно, что светочувствительный пигмент йодопсин находящийся во всех колбочках глаза, включает в себя такие пигменты, как хлоролаб и эритролаб. Оба эти пигмента чувствительны ко всей области видимого спектра, однако первый из них имеет максимум поглощения, соответствующий жёлто-зеленой (максимум поглощения около 540 нм.), а второй жёлто-красной (оранжевой) (максимум поглощения около 570 нм.) частям спектра. Обращает на себя внимание тот факт, что их максимумы поглощения расположены рядом. Это не соответствуют принятым «основным» цветам и не согласуется с основными принципами трёхкомпонентной модели.


Третий, гипотетический пигмент, чувствительный к фиолетово-синей области спектра, заранее получивший название цианолаб, на сегодняшний день так и не найден.


Кроме того, найти какую-либо разницу между колбочками в сетчатке глаза не удалось, не удалось и доказать наличие в каждой колбочке только одного типа пигмента. Более того, было признано, что в колбочке одновременно находятся пигменты хлоролаб и эритролаб.
\begin{figure}[h]
\center{\includegraphics[width=1\linewidth]{Cone-response}}
\caption{Кривые спектров поглощения пигментов содержащихся в колбочках и палочках сетчатки глаза человека. Спектры коротких (S), средних (М) и длинноволновых (L) пигментов и спектр пигмента палочки при слабом (сумеречном) освещении (R).}
\label{ris:image}
\end{figure}

\end{document}